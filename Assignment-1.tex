% Options for packages loaded elsewhere
\PassOptionsToPackage{unicode}{hyperref}
\PassOptionsToPackage{hyphens}{url}
%
\documentclass[
]{article}
\usepackage{amsmath,amssymb}
\usepackage{iftex}
\ifPDFTeX
  \usepackage[T1]{fontenc}
  \usepackage[utf8]{inputenc}
  \usepackage{textcomp} % provide euro and other symbols
\else % if luatex or xetex
  \usepackage{unicode-math} % this also loads fontspec
  \defaultfontfeatures{Scale=MatchLowercase}
  \defaultfontfeatures[\rmfamily]{Ligatures=TeX,Scale=1}
\fi
\usepackage{lmodern}
\ifPDFTeX\else
  % xetex/luatex font selection
\fi
% Use upquote if available, for straight quotes in verbatim environments
\IfFileExists{upquote.sty}{\usepackage{upquote}}{}
\IfFileExists{microtype.sty}{% use microtype if available
  \usepackage[]{microtype}
  \UseMicrotypeSet[protrusion]{basicmath} % disable protrusion for tt fonts
}{}
\makeatletter
\@ifundefined{KOMAClassName}{% if non-KOMA class
  \IfFileExists{parskip.sty}{%
    \usepackage{parskip}
  }{% else
    \setlength{\parindent}{0pt}
    \setlength{\parskip}{6pt plus 2pt minus 1pt}}
}{% if KOMA class
  \KOMAoptions{parskip=half}}
\makeatother
\usepackage{xcolor}
\usepackage[margin=1in]{geometry}
\usepackage{color}
\usepackage{fancyvrb}
\newcommand{\VerbBar}{|}
\newcommand{\VERB}{\Verb[commandchars=\\\{\}]}
\DefineVerbatimEnvironment{Highlighting}{Verbatim}{commandchars=\\\{\}}
% Add ',fontsize=\small' for more characters per line
\usepackage{framed}
\definecolor{shadecolor}{RGB}{248,248,248}
\newenvironment{Shaded}{\begin{snugshade}}{\end{snugshade}}
\newcommand{\AlertTok}[1]{\textcolor[rgb]{0.94,0.16,0.16}{#1}}
\newcommand{\AnnotationTok}[1]{\textcolor[rgb]{0.56,0.35,0.01}{\textbf{\textit{#1}}}}
\newcommand{\AttributeTok}[1]{\textcolor[rgb]{0.13,0.29,0.53}{#1}}
\newcommand{\BaseNTok}[1]{\textcolor[rgb]{0.00,0.00,0.81}{#1}}
\newcommand{\BuiltInTok}[1]{#1}
\newcommand{\CharTok}[1]{\textcolor[rgb]{0.31,0.60,0.02}{#1}}
\newcommand{\CommentTok}[1]{\textcolor[rgb]{0.56,0.35,0.01}{\textit{#1}}}
\newcommand{\CommentVarTok}[1]{\textcolor[rgb]{0.56,0.35,0.01}{\textbf{\textit{#1}}}}
\newcommand{\ConstantTok}[1]{\textcolor[rgb]{0.56,0.35,0.01}{#1}}
\newcommand{\ControlFlowTok}[1]{\textcolor[rgb]{0.13,0.29,0.53}{\textbf{#1}}}
\newcommand{\DataTypeTok}[1]{\textcolor[rgb]{0.13,0.29,0.53}{#1}}
\newcommand{\DecValTok}[1]{\textcolor[rgb]{0.00,0.00,0.81}{#1}}
\newcommand{\DocumentationTok}[1]{\textcolor[rgb]{0.56,0.35,0.01}{\textbf{\textit{#1}}}}
\newcommand{\ErrorTok}[1]{\textcolor[rgb]{0.64,0.00,0.00}{\textbf{#1}}}
\newcommand{\ExtensionTok}[1]{#1}
\newcommand{\FloatTok}[1]{\textcolor[rgb]{0.00,0.00,0.81}{#1}}
\newcommand{\FunctionTok}[1]{\textcolor[rgb]{0.13,0.29,0.53}{\textbf{#1}}}
\newcommand{\ImportTok}[1]{#1}
\newcommand{\InformationTok}[1]{\textcolor[rgb]{0.56,0.35,0.01}{\textbf{\textit{#1}}}}
\newcommand{\KeywordTok}[1]{\textcolor[rgb]{0.13,0.29,0.53}{\textbf{#1}}}
\newcommand{\NormalTok}[1]{#1}
\newcommand{\OperatorTok}[1]{\textcolor[rgb]{0.81,0.36,0.00}{\textbf{#1}}}
\newcommand{\OtherTok}[1]{\textcolor[rgb]{0.56,0.35,0.01}{#1}}
\newcommand{\PreprocessorTok}[1]{\textcolor[rgb]{0.56,0.35,0.01}{\textit{#1}}}
\newcommand{\RegionMarkerTok}[1]{#1}
\newcommand{\SpecialCharTok}[1]{\textcolor[rgb]{0.81,0.36,0.00}{\textbf{#1}}}
\newcommand{\SpecialStringTok}[1]{\textcolor[rgb]{0.31,0.60,0.02}{#1}}
\newcommand{\StringTok}[1]{\textcolor[rgb]{0.31,0.60,0.02}{#1}}
\newcommand{\VariableTok}[1]{\textcolor[rgb]{0.00,0.00,0.00}{#1}}
\newcommand{\VerbatimStringTok}[1]{\textcolor[rgb]{0.31,0.60,0.02}{#1}}
\newcommand{\WarningTok}[1]{\textcolor[rgb]{0.56,0.35,0.01}{\textbf{\textit{#1}}}}
\usepackage{longtable,booktabs,array}
\usepackage{calc} % for calculating minipage widths
% Correct order of tables after \paragraph or \subparagraph
\usepackage{etoolbox}
\makeatletter
\patchcmd\longtable{\par}{\if@noskipsec\mbox{}\fi\par}{}{}
\makeatother
% Allow footnotes in longtable head/foot
\IfFileExists{footnotehyper.sty}{\usepackage{footnotehyper}}{\usepackage{footnote}}
\makesavenoteenv{longtable}
\usepackage{graphicx}
\makeatletter
\def\maxwidth{\ifdim\Gin@nat@width>\linewidth\linewidth\else\Gin@nat@width\fi}
\def\maxheight{\ifdim\Gin@nat@height>\textheight\textheight\else\Gin@nat@height\fi}
\makeatother
% Scale images if necessary, so that they will not overflow the page
% margins by default, and it is still possible to overwrite the defaults
% using explicit options in \includegraphics[width, height, ...]{}
\setkeys{Gin}{width=\maxwidth,height=\maxheight,keepaspectratio}
% Set default figure placement to htbp
\makeatletter
\def\fps@figure{htbp}
\makeatother
\setlength{\emergencystretch}{3em} % prevent overfull lines
\providecommand{\tightlist}{%
  \setlength{\itemsep}{0pt}\setlength{\parskip}{0pt}}
\setcounter{secnumdepth}{-\maxdimen} % remove section numbering
\ifLuaTeX
  \usepackage{selnolig}  % disable illegal ligatures
\fi
\IfFileExists{bookmark.sty}{\usepackage{bookmark}}{\usepackage{hyperref}}
\IfFileExists{xurl.sty}{\usepackage{xurl}}{} % add URL line breaks if available
\urlstyle{same}
\hypersetup{
  pdftitle={Mandatory Assignment 1 - STK-IN4300},
  pdfauthor={Alessia Sanfelici},
  hidelinks,
  pdfcreator={LaTeX via pandoc}}

\title{Mandatory Assignment 1 - STK-IN4300}
\author{Alessia Sanfelici}
\date{September 21, 2023}

\begin{document}
\maketitle

The selected dataset is about Data Science Salaries and was found on the
website Kaggle at the following link:
\url{https://www.kaggle.com/datasets/arnabchaki/data-science-salaries-2023}.

The dataset contains 3755 observations and 11 variables. Each row
represents one employee, with all the informations described by the
columns. There are no missing values, meaning that the whole dataset is
full.

The variables are the following:

\begin{itemize}
\item
  \(work\_year\): the year the salary was paid;
\item
  \(experience\_level\): experience level of the employee during the
  year;
\item
  \(employment\_type\): type of employee;
\item
  \(job\_title\): role of the employee during the year;
\item
  \(salary\): total salary expressed in the currency of the country of
  work;
\item
  \(salary\_currency\): currency of the country of work;
\item
  \(salary\_in\_usd\): salary converted in USD;
\item
  \(employee\_residence\): primary country of residence of the employee
  during the working year;
\item
  \(remote\_ratio\): overall amount of work done remotely;
\item
  \(company\_location\): country of the company where the employee
  works;
\item
  \(company\_size\): size (in terms of number of working people) of the
  company during the year;
\end{itemize}

The categorical variables are: \(experience\_level\),
\(employment\_type\), \(job\_title\), \(salary\_currency\),
\(employee\_residence\), \(company\_location\) and \(company\_size\).

The integer variables are: \(salary\) and \(salary\_in\_usd\).

The remaining variables, \(work\_year\) and \(remote\_ratio\), are
integer variables but, since they are not continuous but discrete (they
represent years and percentages respectively), it is better to consider
them as categorical variables.

This dataset could be used for many purposes of analysis. For example,
it could be used to make a prediction of the salary of an employee,
based on his/hers characteristics about the country of residence, the
currency, the company, and so on. Another application of this dataset
could be a clustering of the data, applied with the aim of grouping
similar observations together.

\hypertarget{problem-1.-summary-statistics-table}{%
\subsection{Problem 1. Summary Statistics
Table}\label{problem-1.-summary-statistics-table}}

For an easier understanding of the data, continuous and categorical
variables have been summarized in two separated tables. That is due to
the fact that these two types of variables can be described with
different quantities and information. The summaries of the variables
have been created by using the library \textbf{gtsummary}.

The following table reports information about the continuous variables:
it includes the mean, the standard deviation, the median and the minimum
and maximum values. \newline

\begin{Shaded}
\begin{Highlighting}[]
\FunctionTok{library}\NormalTok{(}\StringTok{"gtsummary"}\NormalTok{)}

\CommentTok{\# Summary for continuous variables}
\NormalTok{df}\SpecialCharTok{\%\textgreater{}\%} 
  \FunctionTok{tbl\_summary}\NormalTok{(}\AttributeTok{type =} \FunctionTok{c}\NormalTok{(salary }\SpecialCharTok{\textasciitilde{}} \StringTok{"continuous2"}\NormalTok{, salary\_in\_usd }\SpecialCharTok{\textasciitilde{}} \StringTok{"continuous2"}\NormalTok{), }
              \AttributeTok{statistic =} \FunctionTok{list}\NormalTok{(}\FunctionTok{all\_continuous}\NormalTok{() }\SpecialCharTok{\textasciitilde{}} \FunctionTok{c}\NormalTok{(}\StringTok{"\{mean\} (\{sd\})"}\NormalTok{, }
                                                    \StringTok{"\{median\}"}\NormalTok{, }\StringTok{"\{min\}{-}\{max\}"}\NormalTok{)),}
              \AttributeTok{digits =} \FunctionTok{all\_continuous}\NormalTok{() }\SpecialCharTok{\textasciitilde{}} \FunctionTok{c}\NormalTok{(}\DecValTok{2}\NormalTok{, }\DecValTok{2}\NormalTok{, }\DecValTok{0}\NormalTok{, }\DecValTok{0}\NormalTok{, }\DecValTok{0}\NormalTok{), }
              \AttributeTok{include =} \FunctionTok{c}\NormalTok{(salary, salary\_in\_usd),}
              \AttributeTok{label =} \FunctionTok{list}\NormalTok{(salary }\SpecialCharTok{\textasciitilde{}} \StringTok{"Salary"}\NormalTok{, salary\_in\_usd }\SpecialCharTok{\textasciitilde{}} \StringTok{"Salary in USD"}\NormalTok{)) }\SpecialCharTok{\%\textgreater{}\%}
  \FunctionTok{bold\_labels}\NormalTok{() }\SpecialCharTok{\%\textgreater{}\%}
  \FunctionTok{modify\_caption}\NormalTok{(}\StringTok{"**Continuous Variables**"}\NormalTok{)}
\end{Highlighting}
\end{Shaded}

\begin{longtable}[]{@{}lc@{}}
\caption{\textbf{Continuous Variables}}\tabularnewline
\toprule\noalign{}
\textbf{Characteristic} & \textbf{N = 3,755} \\
\midrule\noalign{}
\endfirsthead
\toprule\noalign{}
\textbf{Characteristic} & \textbf{N = 3,755} \\
\midrule\noalign{}
\endhead
\bottomrule\noalign{}
\endlastfoot
\textbf{Salary} & \\
Mean (SD) & 190,695.57 (671,676.50) \\
Median & 138,000 \\
Minimum-Maximum & 6,000-30,400,000 \\
\textbf{Salary in USD} & \\
Mean (SD) & 137,570.39 (63,055.63) \\
Median & 135,000 \\
Minimum-Maximum & 5,132-450,000 \\
\end{longtable}

\newline

For the remaining variables, a new table has been created. Since the
variables are categorical, other information needs to be outlighted: the
absolute and the relative frequency of each variable in the dataset.

For the creation of a clearer summary table for categorical variables,
some columns have been removed, since they were characterized by many
different values with very low frequency (otherwise the table would be
infinite). These columns are: \(job\_title\), \(employee\_residence\)
and \(company\_location\).

\begin{Shaded}
\begin{Highlighting}[]
\CommentTok{\# Summary for categorical variables }
\NormalTok{df }\SpecialCharTok{\%\textgreater{}\%}
  \FunctionTok{tbl\_summary}\NormalTok{(}\AttributeTok{type =} \FunctionTok{everything}\NormalTok{() }\SpecialCharTok{\textasciitilde{}} \StringTok{"categorical"}\NormalTok{, }
              \AttributeTok{digits =} \FunctionTok{all\_categorical}\NormalTok{() }\SpecialCharTok{\textasciitilde{}} \FunctionTok{c}\NormalTok{(}\DecValTok{0}\NormalTok{, }\DecValTok{2}\NormalTok{), }
              \AttributeTok{label =} \FunctionTok{list}\NormalTok{(work\_year }\SpecialCharTok{\textasciitilde{}} \StringTok{"Work Year"}\NormalTok{, }
\NormalTok{                           experience\_level }\SpecialCharTok{\textasciitilde{}} \StringTok{"Experience Level"}\NormalTok{,}
\NormalTok{                           employment\_type }\SpecialCharTok{\textasciitilde{}} \StringTok{"Employment type"}\NormalTok{,}
\NormalTok{                           salary\_currency }\SpecialCharTok{\textasciitilde{}} \StringTok{"Salary Currency"}\NormalTok{,}
\NormalTok{                           remote\_ratio }\SpecialCharTok{\textasciitilde{}} \StringTok{"Remote Ratio"}\NormalTok{,}
\NormalTok{                           company\_size }\SpecialCharTok{\textasciitilde{}} \StringTok{"company Size"}\NormalTok{),}
              \AttributeTok{include =} \FunctionTok{c}\NormalTok{(work\_year, experience\_level, employment\_type,}
\NormalTok{                          salary\_currency, remote\_ratio, company\_size),}
              \AttributeTok{statistic =} \FunctionTok{list}\NormalTok{(}\FunctionTok{all\_categorical}\NormalTok{() }\SpecialCharTok{\textasciitilde{}} \StringTok{"\{n\} (\{p\}\%)"}\NormalTok{)) }\SpecialCharTok{\%\textgreater{}\%}
  \FunctionTok{bold\_labels}\NormalTok{() }\SpecialCharTok{\%\textgreater{}\%}
  \FunctionTok{modify\_caption}\NormalTok{(}\StringTok{"**Categorical Variables**"}\NormalTok{)}
\end{Highlighting}
\end{Shaded}

\begin{longtable}[]{@{}lc@{}}
\caption{\textbf{Categorical Variables}}\tabularnewline
\toprule\noalign{}
\textbf{Characteristic} & \textbf{N = 3,755} \\
\midrule\noalign{}
\endfirsthead
\toprule\noalign{}
\textbf{Characteristic} & \textbf{N = 3,755} \\
\midrule\noalign{}
\endhead
\bottomrule\noalign{}
\endlastfoot
\textbf{Work Year} & \\
2020 & 76 (2.02\%) \\
2021 & 230 (6.13\%) \\
2022 & 1,664 (44.31\%) \\
2023 & 1,785 (47.54\%) \\
\textbf{Experience Level} & \\
EN & 320 (8.52\%) \\
EX & 114 (3.04\%) \\
MI & 805 (21.44\%) \\
SE & 2,516 (67.00\%) \\
\textbf{Employment type} & \\
CT & 10 (0.27\%) \\
FL & 10 (0.27\%) \\
FT & 3,718 (99.01\%) \\
PT & 17 (0.45\%) \\
\textbf{Salary Currency} & \\
AUD & 9 (0.24\%) \\
BRL & 6 (0.16\%) \\
CAD & 25 (0.67\%) \\
CHF & 4 (0.11\%) \\
CLP & 1 (0.03\%) \\
CZK & 1 (0.03\%) \\
DKK & 3 (0.08\%) \\
EUR & 236 (6.28\%) \\
GBP & 161 (4.29\%) \\
HKD & 1 (0.03\%) \\
HUF & 3 (0.08\%) \\
ILS & 1 (0.03\%) \\
INR & 60 (1.60\%) \\
JPY & 3 (0.08\%) \\
MXN & 1 (0.03\%) \\
PLN & 5 (0.13\%) \\
SGD & 6 (0.16\%) \\
THB & 2 (0.05\%) \\
TRY & 3 (0.08\%) \\
USD & 3,224 (85.86\%) \\
\textbf{Remote Ratio} & \\
0 & 1,923 (51.21\%) \\
50 & 189 (5.03\%) \\
100 & 1,643 (43.75\%) \\
\textbf{company Size} & \\
L & 454 (12.09\%) \\
M & 3,153 (83.97\%) \\
S & 148 (3.94\%) \\
\end{longtable}

\hypertarget{problem-2.-bad-data-visualization}{%
\subsection{Problem 2. Bad Data
Visualization}\label{problem-2.-bad-data-visualization}}

The first graph represents the average salary (in USD) of the employees
with respect to the work year and also to the size of the company. Using
a bar plot with stacked bars allows us to have an idea of the amount of
average salary for each year, but it is difficult to make a comparison
between the different values of the salary with respect to year and
company size. That is, for each year, we can only see the total salary,
obtained by the sum of the average salaries for small, medium and large
companies, without any indication about the single values, which cannot
be compared. \newline

\begin{Shaded}
\begin{Highlighting}[]
\CommentTok{\# Bad plot 1}
\NormalTok{by\_year }\OtherTok{\textless{}{-}} \FunctionTok{group\_by}\NormalTok{(df, work\_year, company\_size)}
\NormalTok{new\_df }\OtherTok{\textless{}{-}} \FunctionTok{summarize}\NormalTok{(by\_year, }\AttributeTok{average\_salary =} \FunctionTok{mean}\NormalTok{(salary\_in\_usd, }\AttributeTok{na.rm =} \ConstantTok{TRUE}\NormalTok{))}

\FunctionTok{ggplot}\NormalTok{(new\_df) }\SpecialCharTok{+}
  \FunctionTok{geom\_bar}\NormalTok{(}\AttributeTok{mapping =} \FunctionTok{aes}\NormalTok{(}\AttributeTok{x =}\NormalTok{ work\_year, }\AttributeTok{y =}\NormalTok{ average\_salary, }\AttributeTok{fill =}\NormalTok{ company\_size), }\AttributeTok{stat =} \StringTok{"identity"}\NormalTok{)}\SpecialCharTok{+}
  \FunctionTok{labs}\NormalTok{(}\AttributeTok{x =} \StringTok{"Work Year"}\NormalTok{, }\AttributeTok{y =} \StringTok{"Average Salary (USD)"}\NormalTok{, }\AttributeTok{title =} \StringTok{"Average Salary by Work Year and Company Size"}\NormalTok{)}
\end{Highlighting}
\end{Shaded}

\includegraphics{Assignment-1_files/figure-latex/unnamed-chunk-3-1.pdf}

The second ``bad'' plot shows the density of the salary according to the
remote ratio. The graph is a barplot, containing bars of different
colors, with respect to the amount of work done remotely (0, 50 or 100).
We can identify three main problems in this plot:

\begin{itemize}
\item
  The bars overlap, making the trends confusing and difficult to
  understand;
\item
  Since the 50 category has a very low number of cases, the
  corresponding bars are very small, making it very hard to catch the
  frequency and the behaviour of this category;
\item
  The values on the X-axis are not so easy to interpret.
\end{itemize}

\begin{Shaded}
\begin{Highlighting}[]
\CommentTok{\# Bad plot 2}
\FunctionTok{ggplot}\NormalTok{(df, }\FunctionTok{aes}\NormalTok{(}\AttributeTok{x =}\NormalTok{ salary\_in\_usd, }\AttributeTok{fill =} \FunctionTok{as.factor}\NormalTok{(remote\_ratio)))}\SpecialCharTok{+}
  \FunctionTok{geom\_histogram}\NormalTok{( }\AttributeTok{color=}\StringTok{\textquotesingle{}\#e9ecef\textquotesingle{}}\NormalTok{, }\AttributeTok{alpha=}\FloatTok{0.6}\NormalTok{, }\AttributeTok{position=}\StringTok{\textquotesingle{}identity\textquotesingle{}}\NormalTok{)}\SpecialCharTok{+}
  \FunctionTok{labs}\NormalTok{(}\AttributeTok{x =} \StringTok{"Salary in USD"}\NormalTok{, }\AttributeTok{y =} \StringTok{"Count"}\NormalTok{, }\AttributeTok{title =} \StringTok{"Frequency of salary according to remote ratio"}\NormalTok{)}
\end{Highlighting}
\end{Shaded}

\includegraphics{Assignment-1_files/figure-latex/unnamed-chunk-4-1.pdf}

\hypertarget{problem-3.-good-data-visualization}{%
\subsection{Problem 3. Good Data
Visualization}\label{problem-3.-good-data-visualization}}

The ``good'' version of the first graph uses a bar for each category of
the company size. Thanks to this trick, it is possible to visualize data
in a more organized way: for each year, we have 3 columns, each one
representing the average salary for each company size. This allows a
deeper analysis, with a particular interest on the behaviour of the
average salary over time, according to the size of the company. For
instance, it is possible to notice that, while small companies had more
or less stable salaries over time, the average salary in large companies
increased with the years, with a growth of more than 250000 USD from
2020 to 2023. \newline

\begin{Shaded}
\begin{Highlighting}[]
\NormalTok{new\_df }\OtherTok{\textless{}{-}} \FunctionTok{summarize}\NormalTok{(by\_year, }\AttributeTok{average\_salary =} \FunctionTok{mean}\NormalTok{(salary\_in\_usd, }\AttributeTok{na.rm =} \ConstantTok{TRUE}\NormalTok{))}

\CommentTok{\# Good plot 1}
\FunctionTok{ggplot}\NormalTok{(new\_df) }\SpecialCharTok{+}
  \FunctionTok{geom\_bar}\NormalTok{(}\AttributeTok{mapping =} \FunctionTok{aes}\NormalTok{(}\AttributeTok{x =}\NormalTok{ work\_year, }\AttributeTok{y =}\NormalTok{ average\_salary, }\AttributeTok{fill =}\NormalTok{ company\_size), }
           \AttributeTok{stat =} \StringTok{"summary"}\NormalTok{, }\AttributeTok{position =} \StringTok{"dodge"}\NormalTok{) }\SpecialCharTok{+} 
  \FunctionTok{labs}\NormalTok{(}\AttributeTok{x =} \StringTok{"Work Year"}\NormalTok{, }\AttributeTok{y =} \StringTok{"Average Salary (USD)"}\NormalTok{, }\AttributeTok{title =} \StringTok{"Average Salary by Work Year and Company Size"}\NormalTok{)}
\end{Highlighting}
\end{Shaded}

\includegraphics{Assignment-1_files/figure-latex/unnamed-chunk-5-1.pdf}

\newline

The second plot have been improved by substituting the histogram with a
density plot. This solution allows to obtain a clearer graph, composed
of just three lines. Each line represents the density distribution of
the salary, according to the remote ratio category. Moreover, since the
data for all the categories have been transformed to densities, we don't
have small bars for a single category anymore. The problem of the values
on the X-axis has been solved by changing the unit of measure of the
axis (from USD to thousand of USD).

According to this graph, for example, it can be understood that
employees with 50\% of remote ratio tend to have a lower salary than the
other categories (the pick of the density is under 100000 USD, probably
around 80000 USD). On the contrary, the other two categories have
similar densities, with a mean around 150000 USD. \newline

\begin{Shaded}
\begin{Highlighting}[]
\CommentTok{\#Good plot 2}

\NormalTok{rem\_0 }\OtherTok{\textless{}{-}} \FunctionTok{subset}\NormalTok{(df, df}\SpecialCharTok{$}\NormalTok{remote\_ratio }\SpecialCharTok{==} \DecValTok{0}\NormalTok{, }\AttributeTok{select =} \FunctionTok{c}\NormalTok{(salary\_in\_usd))}
\NormalTok{rem\_50 }\OtherTok{\textless{}{-}} \FunctionTok{subset}\NormalTok{(df, df}\SpecialCharTok{$}\NormalTok{remote\_ratio }\SpecialCharTok{==} \DecValTok{50}\NormalTok{, }\AttributeTok{select =} \FunctionTok{c}\NormalTok{(salary\_in\_usd))}
\NormalTok{rem\_100 }\OtherTok{\textless{}{-}} \FunctionTok{subset}\NormalTok{(df, df}\SpecialCharTok{$}\NormalTok{remote\_ratio }\SpecialCharTok{==} \DecValTok{100}\NormalTok{, }\AttributeTok{select =} \FunctionTok{c}\NormalTok{(salary\_in\_usd))}
\NormalTok{d0 }\OtherTok{\textless{}{-}} \FunctionTok{density}\NormalTok{(}\FunctionTok{as.vector}\NormalTok{(rem\_0[[}\StringTok{"salary\_in\_usd"}\NormalTok{]]))}
\NormalTok{d50 }\OtherTok{\textless{}{-}} \FunctionTok{density}\NormalTok{(}\FunctionTok{as.vector}\NormalTok{(rem\_50[[}\StringTok{"salary\_in\_usd"}\NormalTok{]]))}
\NormalTok{d100 }\OtherTok{\textless{}{-}} \FunctionTok{density}\NormalTok{(}\FunctionTok{as.vector}\NormalTok{(rem\_100[[}\StringTok{"salary\_in\_usd"}\NormalTok{]]))}
\FunctionTok{plot}\NormalTok{(d50, }\AttributeTok{col =} \StringTok{"red"}\NormalTok{, ,}\AttributeTok{main =} \StringTok{"Density of salary according to remote ratio"}\NormalTok{,}
     \AttributeTok{xlab =} \StringTok{"Salary in thousands of USD"}\NormalTok{, }\AttributeTok{ylab =} \StringTok{"Density"}\NormalTok{, }\AttributeTok{xaxt =} \StringTok{"n"}\NormalTok{) }
\CommentTok{\# Customize the x{-}axis labels to be in thousands}
\FunctionTok{axis}\NormalTok{(}\DecValTok{1}\NormalTok{, }\AttributeTok{at =} \FunctionTok{seq}\NormalTok{(}\FloatTok{1e5}\NormalTok{, }\FloatTok{5e5}\NormalTok{, }\AttributeTok{by =} \FloatTok{1e5}\NormalTok{),}
     \AttributeTok{labels =}\NormalTok{ scales}\SpecialCharTok{::}\FunctionTok{comma\_format}\NormalTok{(}\AttributeTok{scale =} \FloatTok{1e{-}3}\NormalTok{)(}\FunctionTok{seq}\NormalTok{(}\FloatTok{1e5}\NormalTok{, }\FloatTok{5e5}\NormalTok{, }\AttributeTok{by =} \FloatTok{1e5}\NormalTok{)))}
\FunctionTok{lines}\NormalTok{(d0, }\AttributeTok{col =} \StringTok{"blue"}\NormalTok{)}
\FunctionTok{lines}\NormalTok{(d100, }\AttributeTok{col =} \StringTok{"green"}\NormalTok{)}
\FunctionTok{legend}\NormalTok{(}\StringTok{"topright"}\NormalTok{, }\AttributeTok{legend =} \FunctionTok{c}\NormalTok{(}\StringTok{"Remote ratio = 0"}\NormalTok{, }\StringTok{"Remote ratio = 50"}\NormalTok{, }\StringTok{"Remote ratio = 100"}\NormalTok{),}
       \AttributeTok{col =} \FunctionTok{c}\NormalTok{(}\StringTok{"blue"}\NormalTok{, }\StringTok{"red"}\NormalTok{, }\StringTok{"green"}\NormalTok{), }\AttributeTok{lty =} \DecValTok{1}\NormalTok{)}
\end{Highlighting}
\end{Shaded}

\includegraphics{Assignment-1_files/figure-latex/unnamed-chunk-6-1.pdf}

\end{document}
